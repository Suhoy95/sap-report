%% Преамбула TeX-файла

% 1. Стиль и язык
\documentclass[utf8x]{G7-32} % Стиль (по умолчанию будет 14pt)
\usepackage[T2A]{fontenc}
\usepackage[russian]{babel}
% Остальные стандартные настройки убраны в preamble.inc.tex.
\include{preamble.inc}

% Настройки листингов.
\include{listings.inc}

% Полезные макросы листингов.
% Любимые команды

\newtheorem{theorem}{Теорема}
\newtheorem{definition}{Определение}

\newcommand{\Code}[1]{\textbf{#1}}


\newcommand{\myImage}[3]{
\begin{figure}[!ht]
    \centering
    \includegraphics[width=0.8\textwidth]{figures/#2}
    \caption{#1}
    \label{#3}
\end{figure}
}


\begin{document}

\pagestyle{empty}
\begin{center}
    Министерство образования и науки Российской Федерации\\
    ФГАОУ ВПО  «УрФУ имени первого Президента России Б. Н. Ельцина»\\
    Институт радиоэлектроники и информационных технологий - РтФ\\
    Департамент информационных технологий и автоматики
    \par
    \vspace{4.5cm}
    \Large{
      Проектирование корпоративных информационных систем (КИС)

      \par
      \vspace{0.5cm}

      ОТЧЕТ\\
      по лабораторной работе
    }

    \vspace{4cm}
    {
      Преподаватель: \hfill Клебанов Борис Исаевич
    }
    \par
    {
      Студент: \hfill Сухоплюев Илья Владимирович
    }
    \par
    {
      Группа: \hfill РИ-440001
    }

    \par
    \vspace{3.5cm}
    Екатеринбург\\
    2017
\end{center}


\frontmatter % выключает нумерацию ВСЕГО; здесь начинаются ненумерованные главы: реферат, введение, глоссарий, сокращения и прочее.

% Команды \breakingbeforechapters и \nonbreakingbeforechapters
% управляют разрывом страницы перед главами.
% По-умолчанию страница разрывается.

% \nobreakingbeforechapters
% \breakingbeforechapters

% \include{00-abstract}
\pagestyle{plain}

\tableofcontents

% \include{10-defines}
% \include{11-abbrev}

\Introduction

Задача создания корпоративных информационных систем давно является рапространенной,
и на овнове этого создаются обобщенные шаблоные решения, которые могут помочь
в быстром конфигурировании и развертовании данной системы на конкретном предприятии,
исходя из его специфических потребностей. Одной из таких платформ,
распространенных в настоящее время является \textit{SAP ERP} (Enterprise Resource Planning),
немецкой корпорации програмного обепечения \textit{SAP}.

Данная ERP-платформа прошла большой путь развития с 1990-ых годов, накопив большой
набор модулей для разнообразных ситуаций, возникающих на предприятии, а так же
огромный опыт по внедреннию данной системы. Полученный опыт по развертыванию,
привел компанию SAP к созданию еще одного продукта -- \textit{SAP Solution Manager},
помогающего быстрее разобраться в построении и внедрении системы SAP на предпрятии.
Данное програмное обеспичение включает в себя набор лучших практик и советов
по настройке SAP-платформы, что позволяет упростить, ускорить, а также избежать
неприятных ошибок, которые могли возникнуть при ручной настройке.

Целью данной лабораторной работы является изучение \textit{SAP Solution Manager}
на примере программы-эмулятора, а также понимание основных этапов построения
корпаротивной информационной системы с его помощью:

\begin{enumerate}
\item Подготовка проекта внедрения;
\item Работа с маршрутными картами;
\item Концептуальный проект;
\item Конфигурация;
\item Организация тестирования;
\item Выполнение тестирования;
\item Анализ результатов тестирования.
\end{enumerate}



\mainmatter % это включает нумерацию глав и секций в документе ниже

\chapter{Подготовка проекта внедрения}
\chapter{Работа с маршрутными картами}
\chapter{Концептуальный проект}
\chapter{Конфигурация}

После того, как концептуальный проект детально описан и утвержен, переходят
к реализации необходимой системы. Первым этапом этого является
конфигурация окружения. Для этого вызываем транзакицю SOLAR02 и получаем
диалоговое окно с аналогичной структурой, как и при построении концептуального проекта:

\myImage{Интерфейс конфигурации проекта}{40-config-start}{40-config-start}

\myImage{После заполнения структуры, во вкладке Коонфигурация можно задать системные параметры проекта и присвоить IMG-обьекты}{30-config}{30-config}

\myImage{И из полученной конфигурации, можно сгенерировать руководство по конфигурации проекта
для тестируемого окружения.}{40-config-doc}{40-config-doc}

\chapter{Тестирование}

Последнем этапом перед развертыванием полученного проекта
является тестирование функционала, который был описан и реализован

\section{Организация тестирования}

Вызвав транзакцию STWB\_2 в SAP Solution Manager мы сможем
определить тесты и группы тестов исходя из созданных процедур и процессов.

\myImage{Создание тестовых сценариев}{50-create}{50-create}
\clearpage
\section{Выполнение тестирования}

\myImage{Назначнение тестовых сценариев тестировщику}{50-assign}{50-assign}

Тестировщик, выполнив транзакцию STWB\_WORK, попадет в интерфейс тестирования,
где сможе протестировать назначенные на его сценарии.
\myImage{Тестирование сценария}{50-work}{50-work}
\clearpage


\section{Анализ результатов тестирования}
И наконец, после проведения всех тестов, мы можем проанализировать
готовность системы к работе и вернуться на предыдущие этапы, если
обнаружидись критические ошибки.
\myImage{Результаты тестирования}{50-results}{50-results}


\backmatter %% Здесь заканчивается нумерованная часть документа и начинаются ссылки и
            %% заключение

\Conclusion % заключение к отчёту

В ходе выполнения лабораторной работы, были изучены этапы внедрения
корпаративной информационной системы, изпользуя платформу SAP Solution Manager.
(Определение требований, создание концептуального проекта, конфигурирование и
тестирование системы).



%%% Local Variables:
%%% mode: latex
%%% TeX-master: "rpz"
%%% End:


% \include{81-biblio}

% \appendix   % Тут идут приложения

% \include{90-appendix1}
% \include{91-appendix2}

\end{document}

%%% Local Variables:
%%% mode: latex
%%% TeX-master: t
%%% End:
