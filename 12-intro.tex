\Introduction

Задача создания корпоративных информационных систем давно является рапространенной,
и на овнове этого создаются обобщенные шаблоные решения, которые могут помочь
в быстром конфигурировании и развертовании данной системы на конкретном предприятии,
исходя из его специфических потребностей. Одной из таких платформ,
распространенных в настоящее время является \textit{SAP ERP} (Enterprise Resource Planning),
немецкой корпорации програмного обепечения \textit{SAP}.

Данная ERP-платформа прошла большой путь развития с 1990-ых годов, накопив большой
набор модулей для разнообразных ситуаций, возникающих на предприятии, а так же
огромный опыт по внедреннию данной системы. Полученный опыт по развертыванию,
привел компанию SAP к созданию еще одного продукта -- \textit{SAP Solution Manager},
помогающего быстрее разобраться в построении и внедрении системы SAP на предпрятии.
Данное програмное обеспичение включает в себя набор лучших практик и советов
по настройке SAP-платформы, что позволяет упростить, ускорить, а также избежать
неприятных ошибок, которые могли возникнуть при ручной настройке.

Целью данной лабораторной работы является изучение \textit{SAP Solution Manager}
на примере программы-эмулятора, а также понимание основных этапов построения
корпаротивной информационной системы с его помощью:

\begin{enumerate}
\item Подготовка проекта внедрения;
\item Работа с маршрутными картами;
\item Концептуальный проект;
\item Конфигурация;
\item Организация тестирования;
\item Выполнение тестирования;
\item Анализ результатов тестирования.
\end{enumerate}

