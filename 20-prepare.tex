\chapter{Подготовка проекта внедрения}

Перед тем, как перейти к разработке и созданию КИС, необходимо собрать
информацию, для которой будет проводится проект внедрения, а также проанализировать
полученную информацию, чтобы наиболее точно определить цели и задачи, которые будет
решать КИС. Для этого необходимо определить следующее:

\begin{enumerate}
\item Определить структуру предприятия (организационные единицы).
\item Определить отрасль, к которой принадлежит предприятие.
\item Определить бизнес-процесс, который следует автоматизировать.
\item Определить IMG–узлы, которые понадобятся для автоматизации выбранного процесса.
\item Определить основные данные об организации, которые понадобятся для автоматизации выбранного процесса.
\end{enumerate}

Рассмотрим подобный проект на примере предприятия <<Техника>>:

\textbf{Техническое задание проекта внедрения <<Excellent>>}
\begin{enumerate}
\item Организационная структура состоит из следующих элементов:
\begin{itemize}
    \item консолидирующая компания (мандант, главная компания);
    \item 2 балансовые единицы (компании); 4 завода;
    \item одна сбытовая организация; канал продаж; сектор; 3 склада.
\end{itemize}
\item Отрасль: High Tech.
\item Бизнес процессом, который следует автоматизировать, является сбыт (модуль сбыт), состоящий из следующих бизнес-шагов (операций):
\begin{itemize}
    \item Создание и Печать заказа клиента;
    \item Создание исходящей поставки;
    \item Отпуск материала;
    \item Создание счёт фактуры;
    \item Просмотр и печать отчета.
\end{itemize}
\end{enumerate}

На основании полученных данных, можно переходить к созданию концептуального проекта
с помощью \textit{SAP Solution Manager}.

% IMG- Implementation Guide