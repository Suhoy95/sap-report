\chapter{Подготовка проекта внедрения}

\section{Предварительна подготовка}

Перед тем, как перейти к разработке и созданию КИС, необходимо собрать
информацию, для которой будет проводится проект внедрения, а также проанализировать
полученную информацию, чтобы наиболее точно определить цели и задачи, которые будет
решать КИС. Для этого необходимо определить следующее:

\begin{enumerate}
\item Определить структуру предприятия (организационные единицы).
\item Определить отрасль, к которой принадлежит предприятие.
\item Определить бизнес-процесс, который следует автоматизировать.
\item Определить IMG–узлы, которые понадобятся для автоматизации выбранного процесса.
\item Определить основные данные об организации, которые понадобятся для автоматизации выбранного процесса.
\end{enumerate}

Рассмотрим подобный проект на примере предприятия <<Техника>>:

\textbf{Техническое задание проекта внедрения <<Excellent>>}
\begin{enumerate}
\item Организационная структура состоит из следующих элементов:
\begin{itemize}
    \item консолидирующая компания (мандант, главная компания);
    \item 2 балансовые единицы (компании); 4 завода;
    \item одна сбытовая организация; канал продаж; сектор; 3 склада.
\end{itemize}
\item Отрасль: High Tech.
\item Бизнес процессом, который следует автоматизировать, является сбыт (модуль сбыт), состоящий из следующих бизнес-шагов (операций):
\begin{itemize}
    \item Создание и Печать заказа клиента;
    \item Создание исходящей поставки;
    \item Отпуск материала;
    \item Создание счёт фактуры;
    \item Просмотр и печать отчета.
\end{itemize}
\end{enumerate}

На основании полученных данных, можно переходить к созданию концептуального проекта
с помощью \textit{SAP Solution Manager}.
% IMG- Implementation Guide

\section{Создание проекта внедрения}
Перейдем к созданию проекта. Для этого вызовем транзакцию SOLAR\_PROJECT\_ADMIN (Управление проектом)
и выберем пункт <<Create project>>, введем имя проекта, а также его тип.

\myImage{Диалоговое окно создания проекта}{20-create-project}{20-create-project}

\subsection*{Вкладка "General data"}

\myImage{После создания, во вкладке "General data" будет предложено заполнить
основные данные о проекте: ответственного за выполнение, язык проекта, запланированные и актуальные даты работы над проектом}{20-general-data}{20-general-data}

\subsection*{Вкладка "Scope"}

\myImage{На вкладке "Scope" мы можем заполнить информацию об области
деятельности нашего проектного решения, за счет определения
маршрутных карт, отрасли предприятия, а также физического рассполажения}{20-scope}{20-scope}

\clearpage
\subsection*{Вкладка "Proj. Team Member"}

\myImage{На данной вкладке мы определяем членов проектной команды,
а также их роли в рамках проекта}{20-proj-team}{20-proj-team}

\subsection*{Вкладка "System Landscape"}

\myImage{На этой вкладке мы задаем "ландшафт" системы: из каких элементов и/или
продуктов будет состоять проект. Логическую структуру проекта. А также пояснения
назначения компонентов в рамках проекта}{20-landscape}{20-landscape}

\subsection*{Вкладка "Milestones"}

\myImage{Далее мы определяем основные этапы по созданию проекта, а также планирование
выполнения этих этапов во времени}{20-milestones}{20-milestones}

\subsection*{Вкладка "Organizational Units"}

\myImage{Здесь мы описываем все структурные единицы (склады, офисы, отделы),
участвующие в организации бизнес процесса}{20-org-units}{20-org-units}