\chapter{Конфигурация}

После того, как концептуальный проект детально описан и утвержен, переходят
к реализации необходимой системы. Первым этапом этого является
конфигурация окружения. Для этого вызываем транзакицю SOLAR02 и получаем
диалоговое окно с аналогичной структурой, как и при построении концептуального проекта:

\myImage{Интерфейс конфигурации проекта}{40-config-start}{40-config-start}

\myImage{После заполнения структуры, во вкладке Коонфигурация можно задать системные параметры проекта и присвоить IMG-обьекты}{30-config}{30-config}

\myImage{И из полученной конфигурации, можно сгенерировать руководство по конфигурации проекта
для тестируемого окружения.}{40-config-doc}{40-config-doc}
