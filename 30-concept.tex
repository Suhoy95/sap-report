\chapter{Концептуальный проект}

После того, как основные сведения о проекте определены и организована проектная группа,
начинается конструирование концептуального проекта, который нужен
для документальной фиксации создаваемой КИС, определения и документирование
бизнес-процессов и системных требований.

Состав концептуального проекта:
\begin{itemize}
\item организационные единицы;
\item основные данные;
\item бизнес-сценарии;
\item бизнес-процессы;
\item шаги процессов.
\end{itemize}

Для того чтобы создать концептуальный проект, нужно выполнить транзакцию
SOLAR01 и выбрать созданный на предыдущем шаге проект. После чего
откроется окно для редактирования проекта:

\myImage{Слева: страктура концептуального проекта. Справа: вкладки для его редактирования}{30-concept-start}{30-concept-start}

\clearpage
\subsection*{Вкладка "Structure"}

\myImage{В вкладке мы задаем основные элементы структуры: будь то
данные компании, бизнес-транзакции или бизнесс-процессы}{30-structure}{30-structure}

\subsection*{Вкладка "Administration"}

\myImage{В "Administration" можно задавать статус, ответственного и время выполняния
того или иного элемента проекта, а также при необхдмости и всех подэлиментов данного элемента}{30-admin}{30-admin}
\clearpage
\subsection*{Вкладка "Transactions"}

\myImage{На данной вкладке происходит присвооение шагам транзаций, программ и обьектов,
что помогает определить шаги процесссов в системе}{30-transaction}{30-transaction}
\clearpage
\subsection*{Вкладка "Proj. Documentation"}

\myImage{На вкладке проектной документации можно создавать и связывать документацию,
используя заготовленные шаблоны, для конкретного элемента проекта}{30-proj-doc}{30-proj-doc}
\clearpage
\subsection*{Вкладка "Graphic"}

\myImage{На данной вкладке можно наглядно увидеть структуру разрабатываемого решения}{30-graphic}{30-graphic}

После описания концептуального проекта, SAP Solution Manager позволяет сгенерировать
документацию проекта для инльного согласования с заказчиком, после чего
она будет учитываться при реализации проекта.
