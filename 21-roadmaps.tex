\chapter{Работа с маршрутными картами}

Одним из важных компонентов программы \textit{SAP SOlution Manager} является
ведение маршрутной карты проекта. Тип маршрутной карты выбирается
исходя из области деятельности разрабатываемого проекта. Маршрутная
карта включает в себя описание основных этапов проекта,
документы (нормативно-организационные, документация, и прочее), а также
помогает организовать группу развертывания по этапам реализации проекта.

Вызвав транзакцию RMMAIN и выбрав наш проект, мы увидим сгенерированную
для него маршрутную карту:

\myImage{Маршрутная карта. Справа -- обозреватель этапов проекта. Внизу -- инструменты для работы
с конкретным этапом. По центру -- область просмотра состояния этапа}{21-roadmap}{21-roadmap}


\myImage{Вкладки приложений маршрутной карты. Accelerator - сгенерированные документы,
ускоряющее разработку проекта. Status/Notes - заметки по ведению проекта. Proj. Team Member -
распределение участников проекта между конкретными этапами. Service Messages - сообщния от
системы об ошибках выполнения на конкретном этапе.Proj. Documentation - документация проекта}{21-roadmap-tabs}{21-roadmap-tabs}